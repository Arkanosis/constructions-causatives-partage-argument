\subsection{Les entrées lexicales}

\begin{figure}[ht]
\centering
\begin{avm}
  [{}
    phon	 & </\emph{jean}/> \\
    synsem &  [{synsem}
	      local & [{loc}
			categorie & [{cat}
				      tete      & nom\\
				      valence   & [{val}
						  SPR   & <>\\
						  SUJ   & <>\\
						  COMPS & <>\\
						  ]\\
				      sous-cat  & <>\\
				    ]\\
			contenu   & [{cont}
				      indice $i$  & [{}
						    personne & 3\textsuperscript{ème}\\
						    nombre   & sing\\
						    genre    & masc\\
						  ]\\
				      restr     & \{ [{}
						      reln  & nomin\\
						      nom	  & jean\\
						      arg	  & $i$\\
						     ]\\
						  \}
				    ]\\
		      ]
	      ]
  ]
\end{avm}
\caption{Entrée lexicale pour \emph{Jean}.\label{lex.jean}}
\end{figure}

\begin{figure}[ht]
\centering
\begin{avm}
  [{}
    phon	 & </\emph{paul}/> \\
    synsem &  [{synsem}
	      local & [{loc}
			categorie & [{cat}
				      tete      & nom\\
				      valence   & [{val}
						  SPR   & <>\\
						  SUJ   & <>\\
						  COMPS & <>\\
						  ]\\
				      sous-cat  & <>\\
				    ]\\
			contenu   & [{cont}
				      indice $i$  & [{}
						    personne & 3\textsuperscript{ème}\\
						    nombre   & sing\\
						    genre    & masc\\
						  ]\\
				      restr     & \{ [{}
						      reln  & nomin\\
						      nom	  & paul\\
						      arg	  & $i$\\
						     ]\\
						  \}
				    ]\\
		      ]
	      ]
  ]
\end{avm}
\caption{Entrée lexicale pour \emph{Paul}.\label{lex.paul}}
\end{figure}

\begin{figure}[ht]
\centering
\begin{avm}
  [{}
    phon	 & </\emph{enfant}/> \\
    synsem &  [{synsem}
	      local & [{loc}
			categorie & [{cat}
				      tete      & [{tete}
				      		  PART & nom]\\
				      valence   & [{val}
						  SPR   & <@{1}>\\
						  SUJ   & <>\\
						  COMPS & <@{2}>\\
						  ]\\
				      sous-cat  & <(@{2})>\\
				    ]\\
			contenu   & [{cont}
				      indice $i$  & [{}
						    personne & 3\textsuperscript{ème}\\
						    nombre   & sing\\
						    genre    & masc\\
						  ]\\
				      restr     & \{ [{}
						      relation  & enfant\\
						      arg	  & $i$\\
						     ]\\
						  \}
				    ]\\
		      ]
	      ]
  ]
\end{avm}
\caption{Entrée lexicale pour \emph{enfant}.\label{lex.enfant}}
\end{figure}

\begin{figure}[ht]
\centering
\begin{avm}
  [{}
    phon	 & </\emph{fils}/> \\
    synsem &  [{synsem}
	      local & [{loc}
			categorie & [{cat}
				      tete      & [{tete}
				      		  PART & nom]\\
				      valence   & [{val}
						  SPR   & <@{1}>\\
						  SUJ   & <>\\
						  COMPS & <@{2}>\\
						  ]\\
				      sous-cat  & <(@{2})>\\
				    ]\\
			contenu   & [{cont}
				      indice $i$  & [{}
						    personne & 3\textsuperscript{ème}\\
						    nombre   & sing\\
						    genre    & masc\\
						  ]\\
				      restr     & \{ [{}
						      relation  & fils\\
						      arg	  & $i$\\
						     ]\\
						  \}
				    ]\\
		      ]
	      ]
  ]
\end{avm}
\caption{Entrée lexicale pour \emph{fils}.\label{lex.fils}}
\end{figure}

\begin{figure}[ht]
\centering
\begin{avm}
  [{}
    phon	 & </\emph{livre}/> \\
    synsem &  [{synsem}
	      local & [{loc}
			categorie & [{cat}
				      tete      & [{tete}
				      		  PART & nom]\\
				      valence   & [{val}
						  SPR   & <@{1}>\\
						  SUJ   & <>\\
						  COMPS & <@{2}>\\
						  ]\\
				      sous-cat  & <(@{2})>\\
				    ]\\
			contenu   & [{cont}
				      indice $i$  & [{}
						    personne & 3\textsuperscript{ème}\\
						    nombre   & sing\\
						    genre    & masc\\
						  ]\\
				      restr     & \{ [{}
						      relation  & livre\\
						      arg	  & $i$\\
						     ]\\
						  \}
				    ]\\
		      ]
	      ]
  ]
\end{avm}
\caption{Entrée lexicale pour \emph{livre}.\label{lex.livre}}
\end{figure}

\begin{figure}[ht]
\centering
\begin{avm}
  [{}
    phon	 & </\emph{roman}/> \\
    synsem &  [{synsem}
	      local & [{loc}
			categorie & [{cat}
				      tete      & [{tete}
				      		  PART & nom]\\
				      valence   & [{val}
						  SPR   & <@{1}>\\
						  SUJ   & <>\\
						  COMPS & <@{2}>\\
						  ]\\
				      sous-cat  & <(@{2})>\\
				    ]\\
			contenu   & [{cont}
				      indice $i$  & [{}
						    personne & 3\textsuperscript{ème}\\
						    nombre   & sing\\
						    genre    & masc\\
						  ]\\
				      restr     & \{ [{}
						      relation  & roman\\
						      arg	  & $i$\\
						     ]\\
						  \}
				    ]\\
		      ]
	      ]
  ]
\end{avm}
\caption{Entrée lexicale pour \emph{roman}.\label{lex.roman}}
\end{figure}

\begin{figure}[ht]
\centering
\begin{avm}
  [{}
    phon	 & </\emph{fait}/> \\
    synsem &  [{synsem}
	      local & [{loc}
			categorie & [{cat}
				      tete      & [{}
						    verbe\\
						    mode  & indicatif
						  ]\\
				      valence   & [{val}
						  suj   & <SN @{1}>\\
						  comps & <@{2}, @{3}, @{4}>\\
						  ]\\
				      sous-cat  & <SN @{1}, V @{4} [{}
								trans	  & -\\
								valence	  & [{val}
									      suj   & <SN @{3}>\\
									      comps & @{2}\\
									    ]\\
								sous-cat  & @{2}
							      ], SN\[\emph{acc}\]@{3} >{} $\bigoplus$ @{2} \\
				    ]\\
			contenu   & [{cont}
				      indice    & s\\
				      restr     & \{ [{}
						      reln	& cause\\
						      agent	& i\\
						      patient	& j\\
						      resultat	& @{1}
						     ]\\
						  \}
				    ]\\
		      ]
	      ]
  ]
\end{avm}
\caption{Entrée lexicale pour \emph{faire} avec complément intransitif.\label{lex.fairei}}
\end{figure}

\begin{figure}[ht]
\centering
\begin{avm}
  [{}
    phon	 & </\emph{fait}/> \\
    synsem &  [{synsem}
	      local & [{loc}
			categorie & [{cat}
				      tete      & [{}
						    verbe\\
						    mode  & indicatif
						  ]\\
				      valence   & [{val}
						  suj   & <SN @{1}>\\
						  comps & <@{2}, @{3}, @{4}>\\
						  ]\\
				      sous-cat  & <SN @{1}, V @{4} [{}
								trans	  & +\\
								valence	  & [{val}
									      suj   & <SN @{3}>\\
									      comps & @{2}\\
									    ]\\
								sous-cat  & @{2}
							      ], SN\[\emph{dat}\]@{3} >{} $\bigoplus$ @{2} \\
				    ]\\
			contenu   & [{cont}
				      indice    & s\\
				      restr     & \{ [{}
						      reln	& cause\\
						      agent	& i\\
						      patient	& j\\
						      resultat	& @{1}
						     ]\\
						  \}
				    ]\\
		      ]
	      ]
  ]
\end{avm}
\caption{Entrée lexicale pour \emph{faire} avec complément transitif.\label{lex.fairet}}
\end{figure}

\begin{figure}[ht]
\centering
\begin{avm}
  [{}
    phon	 & </\emph{dormir}/> \\
    synsem &  [{synsem}
	      local & [{loc}
			categorie & [{cat}
				      tete      & [{}
						    verbe\\
						    mode  & infinitif
						  ]\\
				      valence   & [{val}
						  suj   & <@{1}>\\
						  comps & <>\\
						  ]\\
				      sous-cat  & <SN$_{i}$ @{1}>\\
				    ]\\
			contenu   & [{cont}
				      indice    & s\\
				      restr     & \{ [{}
						      reln	& dormir\\
						      dormeur	& $i$
						     ]\\
						  \}
				    ]\\
		      ]
	      ]
  ]
\end{avm}
\caption{Entrée lexicale pour \emph{dormir}\label{lex.dormir}}
\end{figure}

% TODO l'
% TODO enfant
% TODO le (pronom)

\begin{figure}[ht]
\centering
\begin{avm}
  [{}
    phon	 & </\emph{lire}/> \\
    synsem &  [{synsem}
	      local & [{loc}
			categorie & [{cat}
				      tete      & [{}
						    verbe\\
						    mode  & infinitif
						  ]\\
				      valence   & [{val}
						  suj   & <@{1}>\\
						  comps & <@{2}>\\
						  ]\\
				      sous-cat  & <SN$_{i}$ @{1}, SN\[\emph{acc}\]$_{j}$ @{2}> \\
				    ]\\
			contenu   & [{cont}
				      indice    & s\\
				      restr     & \{ [{}
						      reln	& lire\\
						      agent	& $i$\\
						      objet	& $j$
						     ]\\
						  \}
				    ]\\
		      ]
	      ]
  ]
\end{avm}
\caption{Entrée lexicale pour \emph{lire}.\label{lex.lire}}
\end{figure}

\begin{figure}[ht]
\centering
\begin{avm}
  [{}
    phon	 & </\emph{traduire}/> \\
    synsem &  [{synsem}
	      local & [{loc}
			categorie & [{cat}
				      tete      & [{}
						    verbe\\
						    mode  & infinitif
						  ]\\
				      valence   & [{val}
						  suj   & <@{1}>\\
						  comps & <@{2}>\\
						  ]\\
				      sous-cat  & <SN$_{i}$ @{1}, SN\[\emph{acc}\]$_{j}$ @{2}> \\
				    ]\\
			contenu   & [{cont}
				      indice    & s\\
				      restr     & \{ [{}
						      reln	& traduire\\
						      agent	& $i$\\
						      objet	& $j$
						     ]\\
						  \}
				    ]\\
		      ]
	      ]
  ]
\end{avm}
\caption{Entrée lexicale pour \emph{traduire}.\label{lex.traduire}}
\end{figure}

% TODO le (déterminant)
% TODO roman
% TODO à
% TODO son
% TODO fils
% TODO lui
% TODO livre


