\subsection{Les règles lexicales}

\begin{figure}[ht]
\centering
\begin{avm}
  [{}
    phon	 & </\emph{paul}/> \\
    synsem &  [{synsem}
	      local & [{loc}
			categorie & [{cat}
				      tete      & nom\\
				      valence   & [{val}
						  SPR   & <>\\
						  SUJ   & <>\\
						  COMPS & <>\\
						  ]\\
				      sous-cat  & <>\\
				    ]\\
			contenu   & [{cont}
				      indice i  & [{}
						    personne & 3\textsuperscript{ème}\\
						    nombre   & sing\\
						    genre    & masc\\
						  ]\\
				      restr     & \{ [{}
						      reln  & nomin\\
						      nom	  & paul\\
						      arg	  & i\\
						     ]\\
						  \}
				    ]\\
		      ]
	      ]
  ]
\end{avm}
\caption{Entrée lexicale pour \emph{Paul}.\label{lex.paul}}
\end{figure}

\begin{figure}[ht]
\centering
\begin{avm}
  [{}
    phon	 & </\emph{faire}/> \\
    synsem &  [{synsem}
	      local & [{loc}
			categorie & [{cat}
				      tete      & [{}
						    verbe\\
						    mode  & indicatif
						  ]\\
				      valence   & [{val}
						  suj   & <SN @{1}>\\
						  comps & <@{2}, @{3}, @{4}>\\
						  ]\\
				      sous-cat  & <SN @{1}, V @{4} [{}
								trans	  & -\\
								valence	  & [{val}
									      suj   & <SN @{3}>\\
									      comps & @{2}\\
									    ]\\
								sous-cat  & @{2}
							      ], SN[{}\emph{acc}]@{3} >{} $\bigoplus$ @{2} \\
				    ]\\
			contenu   & [{cont}
				      indice    & s\\
				      restr     & \{ [{}
						      reln	& cause\\
						      agent	& i\\
						      patient	& j\\
						      resultat	& @{1}
						     ]\\
						  \}
				    ]\\
		      ]
	      ]
  ]
\end{avm}
\caption{Entrée lexicale pour \emph{faire} avec complément intransitif.\label{lex.paul}}
\end{figure}

