\subsection{Les règles lexicales}

\begin{figure}[ht]
\centering
\begin{avm}
  [{}
    phon         & </\emph{fait}/> \\
    synsem = Cat & [{} Sous-cat = < SN @{1}, V[{} Sous-cat < SN\[\emph{acc}\] @{2}> $\oplus$ @{4} ] @{3} + @{2} > ]
  ]
\end{avm}\\
$\Downarrow$\\
\begin{avm}
  [{}
    phon   & </\emph{le-fait}/> \\
    synsem & [{}
	      Local = Cat & [{} Sous-cat = < SN @{1}, V [{} Sous-cat @{4} ], @{3} > ]\\
	      Non-Local   & [{} CL = { SN\[\emph{acc}\] @{2} } ]\\
             ]\\
  ]
\end{avm}
\caption{Règle lexicale pour la cliticisation avec un pronom COD.\label{regle.cl.cod}}
\end{figure}

\begin{figure}[ht]
\centering
\begin{avm}
  [{}
    phon         & </\emph{fait}/> \\
    synsem = Cat & [{} Sous-cat = < SN\[\emph{dat}\] @{1}, SN @{2} > ]
  ]
$\Rightarrow$
  [{}
    phon   & </\emph{lui-fait}/> \\
    synsem & [{}
	      Local = Cat & [{} Sous-cat = < SN\[\emph{dat}\] @{1} > ]\\
	      Non-Local   & [{} CL = { SN @{2} } ]\\
             ]\\
  ]
\end{avm}
\caption{Règle lexicale pour la cliticisation avec un pronom COI.\label{regle.cl.coi}}
\end{figure}

\begin{figure}[ht]
\centering
\begin{avm}
  [{}
    phon         & </\emph{fait}/> \\
    synsem = Cat & [{} Sous-cat = < SN\[\emph{dat}\] @{1}, SN @{2} > ]
  ]
$\Rightarrow$
  [{}
    phon   & </\emph{le-lui-fait}/> \\
    synsem & [{}
	      Local = Cat & [{} Sous-cat = < SN\[\emph{dat}\] @{1} > ]\\
	      Non-Local   & [{} CL = { SN @{2} } ]\\
             ]\\
  ]
\end{avm}
\caption{Règle lexicale pour la cliticisation avec un propose COD et un pronom COI.\label{regle.cl.cod.coi}}
\end{figure}
