\subsection{Les règles lexicales}

\begin{figure}[ht]
\centering
\begin{avm}
  [{}
    phon         & </\emph{fait}/> \\
    synsem = Cat & [{} Sous-cat = < SN @{1}, V[{}
						TRANS & -\\
						Sous-cat & < SN @{2}> $\oplus$ @{4} ] @{3}, @{2} >
					      ]
  ]
\end{avm}\\
$\Downarrow$\\
\begin{avm}
  [{}
    phon   & </\emph{le-fait}/> \\
    synsem & [{}
	      Local = Cat & [{} Sous-cat = < SN @{1}, V [{}
						          TRANS    & -\\
							  Sous-cat & @{4}
							] @{3} > ]\\
	      Non-Local   & [{} CL = < SN\[\emph{acc}\] @{2} > ]\\
             ]\\
  ]
\end{avm}
\caption{Règle lexicale pour la cliticisation en pronom COD avec un verbe intransitif à l'infinitif.\label{regle.cl.cod}}
\end{figure}

\begin{figure}[ht]
\centering
\begin{avm}
  [{}
    phon         & </\emph{fait}/> \\
    synsem = Cat & [{} Sous-cat = < SN @{1}, V[{}
						TRANS & +\\
						Sous-cat & < SN @{2}, SN\[\emph{acc}\] @{5}> $\oplus$ @{4} ] @{3}, SN\[\emph{acc}\] @{5}, SN\[\emph{dat}\] @{2} >
					      ]
  ]
\end{avm}\\
$\Downarrow$\\
\begin{avm}
  [{}
    phon   & </\emph{le-fait}/> \\
    synsem & [{}
	      Local = Cat & [{} Sous-cat = < SN @{1}, V [{}
						          TRANS    & +\\
							  Sous-cat & < SN\[\emph{acc}\] @{5} > $\oplus$ @{4}
							] @{3}, SN\[\emph{dat}\] @{2} > ]\\
	      Non-Local   & [{} CL = < SN\[\emph{acc}\] @{5} > ]\\
             ]\\
  ]
\end{avm}
\caption{Règle lexicale pour la cliticisation en pronom COD avec un verbe transitif à l'infinitif.\label{regle.cl.cod.trans}}
\end{figure}

\begin{figure}[ht]
\centering
\begin{avm}
  [{}
    phon         & </\emph{fait}/> \\
    synsem = Cat & [{} Sous-cat = < SN @{1}, V[{}
						TRANS & +\\
						Sous-cat & < SN @{2}, SN\[\emph{acc}\] @{5}> $\oplus$ @{4} ] @{3}, SN\[\emph{acc}\] @{5}, SN\[\emph{dat}\] @{2} >
					      ]
  ]
\end{avm}\\
$\Downarrow$\\
\begin{avm}
  [{}
    phon   & </\emph{lui-fait}/> \\
    synsem & [{}
	      Local = Cat & [{} Sous-cat = < SN @{1}, V [{}
						          TRANS    & +\\
							  Sous-cat & < SN\[\emph{acc}\] @{5} > $\oplus$ @{4}
							] @{3}, SN\[\emph{acc}\] @{5} > ]\\
	      Non-Local   & [{} CL = < SN\[\emph{dat}\] @{2} > ]\\
             ]\\
  ]
\end{avm}
\caption{Règle lexicale pour la cliticisation en pronom COI avec un verbe transitif à l'infinitif.\label{regle.cl.coi.trans}}
\end{figure}

\begin{figure}[ht]
\centering
\begin{avm}
  [{}
    phon         & </\emph{fait}/> \\
    synsem = Cat & [{} Sous-cat = < SN @{1}, V[{}
						TRANS & +\\
						Sous-cat & < SN @{2}, SN\[\emph{acc}\] @{5}> $\oplus$ @{4} ] @{3}, SN\[\emph{acc}\] @{5}, SN\[\emph{dat}\] @{2} >
					      ]
  ]
\end{avm}\\
$\Downarrow$\\
\begin{avm}
  [{}
    phon   & </\emph{le-lui-fait}/> \\
    synsem & [{}
	      Local = Cat & [{} Sous-cat = < SN @{1}, V [{}
						          TRANS    & +\\
							  Sous-cat & < SN\[\emph{acc}\] @{5} > $\oplus$ @{4}
							] @{3} > ]\\
	      Non-Local   & [{} CL = < SN\[\emph{acc}\] @{2} , SN\[\emph{dat}\] @{5} > ]\\
             ]\\
  ]
\end{avm}\\
Cette règle gère l'ordre des pronoms clitiques en mettant le pronom COD avant le pronom COI.
\caption{Règle lexicale pour la cliticisation en pronom COD et COI avec un verbe transitif à l'infinitif.\label{regle.cl.cod.coi}}
\end{figure}

\begin{figure}[ht]
\centering
\begin{avm}
  [{}
    phon	 & </\emph{nom}/> \\
    synsem =
	      local =
			categorie & [{cat}
				      tete      & [{tete}
				      		  PART & nom]\\
				      valence   & [{val}
						  SPR   & <@{1}>\\
						  SUJ   & <>\\
						  COMPS & <@{2}>\\
						  ]\\
				      sous-cat  & <@{1}, @{2}>\\
				    ]\\
  ]
\end{avm}\\
$\Downarrow$\\
\begin{avm}
  [{}
    phon	 & </\emph{det}-\emph{nom}/> \\
    synsem =
	      local =
			categorie & [{cat}
				      tete      & [{tete}
				      		  PART & nom]\\
				      valence   & [{val}
						  SPR   & <>\\
						  SUJ   & <>\\
						  COMPS & <@{2}>\\
						  ]\\
				      sous-cat  & <@{2}>\\
				    ]\\
  ]
\end{avm}
\caption{Règles lexicale pour la spécification d'un nom commun.}
\end{figure}

