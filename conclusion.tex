\section{Conclusion}

Grace à la compréhension du phénomène linguistique et des problèmes théoriques posés par la langue, nous avons pu rédiger une portion de grammaire HPSG comprenant les entrées lexicales nécessaires à notre corpus et les règles lexicales qui permettent de gérer les critères de sélection combinatoire.

Nous avons également élaboré des arbres décorés qui donnent à voir la représentation sous forme d'arbre des phrases du corpus que nous souhaitions analyser.\\

Cette analyse nous a permis de prendre conscience de la puissance d'HPSG et de se capacité à gérer des phénomènes de langue
complexes comme les constructions causatives et le partage d'arguments.

Dès lors, on comprend pourquoi HPSG est un formalisme qui commence à être utilisé dans certaines applications du TAL.
En effet, même si HPSG n'est pas aussi utilisé que LFG, par exemple, il a l'avantage de minimiser le noyau de la grammaire et de donner une plus grande importance aux informations lexicales.
