\section{Conclusion}

Grace � la compr�hension du ph�nom�ne linguistique et des probl�mes th�oriques pos�s par la langue, nous avons
pu r�diger une portion de grammaire HPSG comprenant les entr�es lexicales n�cessaires � notre corpus et les r�gles lexicales
qui permettent de g�rer les crit�res de s�lection combinatoire. Nous avons �galement �labor� des arbres d�cor�s qui donnent
� voir la repr�sentation sous forme d'arbre des phrases du corpus que nous souhaitions analyser.

Cette analyse nous a permis de prendre conscience de la puissance d'HPSG et de se capacit� � g�rer des ph�nom�nes de langue
complexes comme les constructions causatives et le partage d'arguments. D�s lors, on comprend pourquoi HPSG est un formalisme
qui commence � �tre utilis� dans certaines applications du TAL. En effet, m�me si HPSG n'est pas aussi utilis� que LFG, par
exemple, il a l'avantage de minimiser le noyau de la grammaire et de donner une plus grande importance aux informations
lexicales.
