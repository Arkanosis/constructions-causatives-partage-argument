\section{Introduction}

Notre expos� va traiter des structures causatives et du partage d'arguments.

Nous tenterons ici de comprendre le ph�nom�ne et de voir comment il peut �tre trait� en utilisant un formalisme particulier : HPSG.

Pour ce faire, nous analyserons dans un premier temps le ph�nom�ne linguistique en tentant de r�pondre aux questions suivantes :
Quels sont les propri�t�s de la langue mises en jeu dans ce ph�nom�ne?
Quels sont les difficult�s empiriques qui se posent � nous ?

Nous verrons aussi certaines propri�t�s g�n�rales de HPSG et d'autres plus particuli�res.
Nous tenterons notamment de voir quels sont les avantages que propose HPSG pour prendre en consid�ration les constructions causatives et le partage d'arguments.
Dans une second temps, nous pr�senterons l'analyse HPSG mise en place pour expliquer le ph�nom�ne.
Pour cela, nous proposerons des structures de traits pour les unit�s lexicales et syntagmatiques, ainsi que des arbres d�cor�s.
