\section{Introduction}

Notre exposé va traiter des structures causatives et du partage d'arguments.

Nous tenterons ici de comprendre le phénomène et de voir comment il peut être traité en utilisant un formalisme particulier : HPSG.

Pour ce faire, nous analyserons dans un premier temps le phénomène linguistique en tentant de répondre aux questions suivantes :
Quels sont les propriétés de la langue mises en jeu dans ce phénomène?
Quels sont les difficultés empiriques qui se posent à nous ?

Nous verrons aussi certaines propriétés générales de HPSG et d'autres plus particulières.
Nous tenterons notamment de voir quels sont les avantages que propose HPSG pour prendre en considération les constructions causatives et le partage d'arguments.
Dans une second temps, nous présenterons l'analyse HPSG mise en place pour expliquer le phénomène.
Pour cela, nous proposerons des structures de traits pour les unités lexicales et syntagmatiques, ainsi que des arbres décorés.

% vim: set fenc=utf-8 ff=unix sw=2 tw=0 :
