Les phrases que nous venons de décrire sont grammaticales et peuvent donc être représentés par la grammaire.
Au contraire, des phrases agrammaticales comme celles que l'on trouve dans le corpus ne peuvent, en principe, pas être représentées par notre grammaire.
Dans cette partie, nous allons tenter de voir par quels principes les phrases agrammaticales sont rejetées.\\

La phrase 11 est agrammaticale car \emph{l'enfant} est placé en position canonique 'sujet' devant le verbe \emph{dormir}.

On sait que les arguments présents dans la structure de traits sont présentés selon leur ordre d'apparition dans les règles.
En plaçant, dans la \autoref{regle.cl.cod} le sujet noté [2] après le verbe noté [3], on empêche la réalisation de phrases
agrammaticales comme la phrase 11.\\

Dans la phrase 13, on remarque que \emph{le roman} est mal placé, mais ici, la première partie de la \autoref{regle.cl.cod.trans} empêche \emph{le roman} de se placer avant \emph{traduire} et le met entre \emph{traduire} et \emph{à son fils}.
En effet, dans cette figure, le verbe noté [3] est placé avant le SN[acc] noté [5] qui est lui même placé avant le SN[dat] noté [2

Cette même figure permet également de rejeter la phrase 14 car comme on le voit dans cette règle, le SN[dat] ne peut pas être placé avant le SN[acc].\\

Les phrases 16 et 17 sont agrammaticales car les formes des pronoms ne correspondent pas aux fonctions des SN d'origine.
Dans la phrase 16, on a le pronom \emph{lui} alors qu'on devrait avoir la forme \emph{le}. C'est la \autoref{regle.cl.cod} qui stipule qu'avec un verbe intransitif, on ne peut trouver qu'un pronom clitique à la forme COD, soit \emph{le} plutôt que \emph{lui}.
La \autoref{regle.cl.coi.trans} permet d'exclure la phrase 17 car si c'est le SN qui est en position COI qui est cliticisé et non pas le SN qui est en position COD (ici \emph{le livre}) alors il est cliticisé sous la forme d'un pronom COI, soit \emph{lui} plutôt que \emph{le}.\\

Les phrases 12 et 15 sont rejetées car le pronom clitique est placé entre le verbe \emph{faire} et le verbe à l'infinitif.
Notre grammaire rejette ces phrases car, comme on l'a expliqué, dans ces cas là on trouve des structures plates plutôt que des
structures hiérarchiques.

Cela permet à \emph{faire} de récupérer les compléments des verbes à l'infinitif, ce que l'on voit dans les règles que nous avons rédigées ci-dessus.
C'est en récupérant les compléments des verbes à l'infinitf que \emph{faire} s'attribue les pronoms qui viennent donc logiquement se placer devant lui.
On postule également l'existence d'une règle générale de HPSG qui précise que les pronoms clitiques (considérés comme des affixes) viennent se positionner avant les verbes qui les sous-catégorisent.
